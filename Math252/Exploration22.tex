\documentclass[12pt,letterpaper,fleqn]{article}

%       amslatex provides nice math extensions for typesetting mathematics
\usepackage{amsmath}
\usepackage{amsfonts}
\usepackage{tmmaths}
\usepackage{sympytex}

%       pstricks provides powerful environments for incorporating postscript into a
%       TeX/LaTeX document. You must have a postscript printer and a package like
%       dvips to convert the DVI file to a PS file.
%\usepackage{pst-all}
%\usepackage{pstricks,pst-plot}
%\usepackage{pst-coil,pst-node}

%  This package provides native tex support for numbered grids. The syntax is:
%  \graphpaper[spc](x_lowleft,y_lowleft)(x_upperright,y_upperright)

%\usepackage{graphpap}
%\usepackage{float}

%  The package below must be initialized with "\initfloatingfigs" immediately after the
%  "\begin{document} command.
%\usepackage{floatfig}

\usepackage{graphicx}
\graphicspath{{i:/mytex/graphics}}
\DeclareGraphicsExtensions{.ps,.eps}

%       tst is a package for the creation of exams, quizzes and tests. the include
%       file mathstuf (see below) provides many abbreviations for these environments.
%\usepackage{tst}

%       epsfig is a package which provides for the inclusion of Encapsulated PostScript
%       files in a document.
%\usepackage{epsfig}
%\usepackage{epic,eepic}
\include{mathstuf}
\usepackage[total={7.25in,10in},top=0.25in,left=0.75in,includehead]{geometry}
\usepackage{fancyhdr}
\pagestyle{fancy}
\lhead{Math 252}
\rhead{\large Name\makebox[2in]{\hrulefill}}
\chead{\LARGE Exploration 22}
%\lfoot{\today}
\cfoot{}
%\rfoot{\thepage}
\renewcommand{\headrulewidth}{0.4pt}
\renewcommand{\footrulewidth}{0.4pt}
\setlength{\parindent}{0pt}
\setlength{\parskip}{2ex}

\newcounter{tf}[enumi]
\newenvironment{tf}[0]{\begin{list}%
{\alph{tf}. \makebox[5em]{True\hfill False}}%
{\usecounter{tf}\setlength{\labelwidth}{7em}%
\setlength{\leftmargin}{3.5cm}%
\setlength{\labelsep}{1cm}}}%
{\end{list}}

%\usepackage{epic,eepic}
\newcommand{\numline}{%
%\newcounter{mark}%
%\setcounter{mark}{-1}%
\setlength{\unitlength}{0.1in}%
\begin{picture}(0,0)%
\thicklines%
\put(0,0){\line(1,0){60}}%
\multiput(0,0)(10,0){7}{\line(0,-1){1}%
\makebox(0,-1.5)[t]{\arabic{mark}}\stepcounter{mark}}%
%
\thinlines%
\multiput(0,0)(5,0){12}{\line(0,-1){0.5}}%
\multiput(0,0)(1,0){60}{\line(0,-1){0.3}}%
%\put(-5,265){\makebox(0,0)[l]{{\bf cm}}}%
\end{picture}}%

\newcommand{\ds}{\displaystyle}
\usepackage{amsfonts}


\let\oldhat\hat
\renewcommand{\hat}[1]{\oldhat{\boldsymbol{\mathbf{#1}}}}
\newcommand{\lv}[1]{\ensuremath{\langle #1 \rangle}}
\renewcommand{\i}{\ensuremath{\hat{\imath}}}
\renewcommand{\j}{\ensuremath{\hat{\jmath}}}
\renewcommand{\k}{\ensuremath{\mathbf{\oldhat{k}}}}
\newcommand{\ora}[1]{\ensuremath{\overrightarrow{#1}}}
\renewcommand{\vec}[1]{\ensuremath{\mathbf{#1}}}
\renewcommand{\v}[1]{\ensuremath{\vec{#1}}}
\newcommand{\abs}[1]{\ensuremath{\lvert #1 \rvert}}

\usepackage{tabularx}
\usepackage{paralist}
\newcommand{\red}[1]{\textcolor{red}{#1}}
\newcommand{\blue}[1]{\textcolor{blue}{#1}}
% \newcommand{\ans}[1]{\quad\fbox{answer: \red{#1}}}
\newcommand{\ans}[1]{\mbox{{\bf Ans:} \blue{#1}}}
\newcommand{\dd}[2][]{\ensuremath{\frac{\text{d}#1}{\text{d}#2}}}
\newcommand{\eval}[2]{\ensuremath{\left.#1\right|_{#2}}}

\usepackage{amsthm}

\theoremstyle{definition}
\newtheorem*{definition}{Definition}

\usepackage{enumitem}
\usepackage{subfig}

\newcommand{\dint}{\ensuremath{\displaystyle\int}}

\begin{document}
\section*{More Improper Integrals}
% The Fundamental Theorem of Calculus allows us to evaluate definite integrals of \emph{continuous functions} defined over \emph{closed, bounded\footnote{Intervals of finite length which include both endpoints.} intervals} (assuming we can find an antiderivative of the integrand).
%
% We can use limiting concepts to extend our notion of the definite integral to:
% \begin{enumerate}
%   \item Continuous functions defined over unbounded intervals of the type:
%   \begin{enumerate}
%     \item $[a, +\infty)$
%     \item $(-\infty, b]$
%     \item $(-\infty, \infty)$
%   \end{enumerate}
%   \item Functions defined over bounded intervals with discontinuities\footnote{Usually infinite discontinuities.} within the interval of integration or at the endpoints of the interval of integration.
% \end{enumerate}
% We do this by picking a variable (called a ``parameter'') within the interval, use this parameter as one endpoint of a closed, bounded interval over which the function is continuous
For some of these problems you might find L'H\^opital's rule helpful to evaluate limits.
\subsection*{L'H\^opital's Rule}
Suppose $f(x)$ and $g(x)$ are functions with either:
\begin{itemize}
  \item $f(x)\to 0$ and $g(x)\to 0$ as $x\to a$, or
  \item $f(x)\to\pm\infty$ and $g(x)\to\pm\infty$ as $x\to a$
\end{itemize}
then
\begin{equation*}
  \lim_{x\to a}\frac{f(x)}{g(x)} = \lim_{x\to a}\frac{f'(x)}{g'(x)}
\end{equation*}
These quotients $f(x)/g(x)$ are called ``indeterminate quotients'' of type $0/0$ and $\infty/\infty$, respectively.

An indeterminate product $f(x)g(x)$ of type $0\cdot\infty$ (where one factor approaches $0$ as $x\to a$ and the other factor approaches $\infty$ as $x\to a$) can be turned into an indeterminate quotient by dividing one factor by the reciprocal of the other.
\subsection*{Problems}
\begin{enumerate}
\item Determine if the integral converges or diverges. If it converges, find its (limiting) value.
\begin{enumerate}
  \item $\ds\int_{-2}^3 \frac{1}{x^4}\;dx$
  \item $\ds\int_1^\infty xe^{-x}\;dx$
  \item $\ds\int_{-\infty}^{\infty}xe^{-x^2}\;dx$
  \item $\ds\int_0^1 x\ln x\;dx$
  \item $\ds\int_0^9 \frac{1}{\sqrt[3]{x-1}}\;dx$
\end{enumerate}
\item Find the values of $a$ for which the integral $\ds\int_0^\infty e^{ax}\cos x\;dx$ converges. Evaluate the integral for those values of $a$.
\end{enumerate}
\newpage
\subsection*{Answers}
\begin{enumerate}
  \item[1a.] Diverges
  \item[1b.] Converges to $2/e$
  \item[1c.] Converges to $0$
  \item[1d.] Converges to $-1/4$
  \item[1e.] Converges to $9/2$
  \item[2.\hspace{0.5em}] Converges to $\dfrac{-a}{1+a^2}$ if $a < 0$. Does not converge if $a\geq 0$.
\end{enumerate}
\end{document}
