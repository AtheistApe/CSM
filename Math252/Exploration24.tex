\documentclass[12pt,letterpaper,fleqn]{article}

%       amslatex provides nice math extensions for typesetting mathematics
\usepackage{amsmath}
\usepackage{amsfonts}
\usepackage{tmmaths}
\usepackage{sympytex}

%       pstricks provides powerful environments for incorporating postscript into a
%       TeX/LaTeX document. You must have a postscript printer and a package like
%       dvips to convert the DVI file to a PS file.
%\usepackage{pst-all}
%\usepackage{pstricks,pst-plot}
%\usepackage{pst-coil,pst-node}

%  This package provides native tex support for numbered grids. The syntax is:
%  \graphpaper[spc](x_lowleft,y_lowleft)(x_upperright,y_upperright)

%\usepackage{graphpap}
%\usepackage{float}

%  The package below must be initialized with "\initfloatingfigs" immediately after the
%  "\begin{document} command.
%\usepackage{floatfig}

\usepackage{graphicx}
\graphicspath{{i:/mytex/graphics}}
\DeclareGraphicsExtensions{.ps,.eps}

%       tst is a package for the creation of exams, quizzes and tests. the include
%       file mathstuf (see below) provides many abbreviations for these environments.
%\usepackage{tst}

%       epsfig is a package which provides for the inclusion of Encapsulated PostScript
%       files in a document.
%\usepackage{epsfig}
%\usepackage{epic,eepic}
\include{mathstuf}
\usepackage[total={7.25in,10in},top=0.25in,left=0.75in,includehead]{geometry}
\usepackage{fancyhdr}
\pagestyle{fancy}
\lhead{Math 252}
\rhead{\large Name\makebox[2in]{\hrulefill}}
\chead{\LARGE Exploration 24}
%\lfoot{\today}
\cfoot{}
%\rfoot{\thepage}
\renewcommand{\headrulewidth}{0.4pt}
\renewcommand{\footrulewidth}{0.4pt}
\setlength{\parindent}{0pt}
\setlength{\parskip}{2ex}

\newcounter{tf}[enumi]
\newenvironment{tf}[0]{\begin{list}%
{\alph{tf}. \makebox[5em]{True\hfill False}}%
{\usecounter{tf}\setlength{\labelwidth}{7em}%
\setlength{\leftmargin}{3.5cm}%
\setlength{\labelsep}{1cm}}}%
{\end{list}}

%\usepackage{epic,eepic}
\newcommand{\numline}{%
%\newcounter{mark}%
%\setcounter{mark}{-1}%
\setlength{\unitlength}{0.1in}%
\begin{picture}(0,0)%
\thicklines%
\put(0,0){\line(1,0){60}}%
\multiput(0,0)(10,0){7}{\line(0,-1){1}%
\makebox(0,-1.5)[t]{\arabic{mark}}\stepcounter{mark}}%
%
\thinlines%
\multiput(0,0)(5,0){12}{\line(0,-1){0.5}}%
\multiput(0,0)(1,0){60}{\line(0,-1){0.3}}%
%\put(-5,265){\makebox(0,0)[l]{{\bf cm}}}%
\end{picture}}%

\newcommand{\ds}{\displaystyle}
\usepackage{amsfonts}


\let\oldhat\hat
\renewcommand{\hat}[1]{\oldhat{\boldsymbol{\mathbf{#1}}}}
\newcommand{\lv}[1]{\ensuremath{\langle #1 \rangle}}
\renewcommand{\i}{\ensuremath{\hat{\imath}}}
\renewcommand{\j}{\ensuremath{\hat{\jmath}}}
\renewcommand{\k}{\ensuremath{\mathbf{\oldhat{k}}}}
\newcommand{\ora}[1]{\ensuremath{\overrightarrow{#1}}}
\renewcommand{\vec}[1]{\ensuremath{\mathbf{#1}}}
\renewcommand{\v}[1]{\ensuremath{\vec{#1}}}
\newcommand{\abs}[1]{\ensuremath{\lvert #1 \rvert}}

\usepackage{tabularx}
\usepackage{paralist}
\newcommand{\red}[1]{\textcolor{red}{#1}}
\newcommand{\blue}[1]{\textcolor{blue}{#1}}
% \newcommand{\ans}[1]{\quad\fbox{answer: \red{#1}}}
\newcommand{\ans}[1]{\mbox{{\bf Ans:} \blue{#1}}}
\newcommand{\dd}[2][]{\ensuremath{\frac{\text{d}#1}{\text{d}#2}}}
\newcommand{\eval}[2]{\ensuremath{\left.#1\right|_{#2}}}

\usepackage{amsthm}

\theoremstyle{definition}
\newtheorem*{definition}{Definition}

\usepackage{enumitem}
\usepackage{subfig}

\newcommand{\dint}{\ensuremath{\displaystyle\int}}

\usepackage{amsthm}
\newtheorem*{theorem}{Theorem}

% \theoremstyle{defintion}
% \newtheorem*{definition}{Definition}

\begin{document}
\section*{Convergent and Divergent Sequences}
A sequence $a_n$ is convergent if there exists a number $L$ such that $\lim_{n\to\infty} a_n = L$. We say that the sequence converges to $L$. If the sequence is not convergent (no such limiting number $L$ exists), we say that the sequence is divergent.

We say a sequence $a_n$ diverges to $\pm\infty$ if $\lim_{n\to\infty} a_n = \pm\infty$.

Since sequences are (discontinuous) functions, we can apply all the limit laws of functions from first semester calculus to sequences\footnote{As long as those laws don't rely on the continuity of the functions.}.

A sequence \emph{eventually} satisfies a condition if there is a position $N$ in the sequence for which the condition is satisfied for all $n > N$.\footnote{The terms of the sequence before position $N$ are the ``head'' of the sequence, those after are the ``tail''; Convergence or divergence depend on what happens in the tail of the sequence.} For example, consider
\theorem[Squeeze Theorem]
  If $a_n\to L$ as $n\to\infty$ and $b_n\to L$ as $n\to\infty$ and \emph{eventually} $a_n\leq c_n\leq b_n$, then $c_n\to L$ as $n\to\infty$.

There are many theorems about convergence that apply only to sequences of (eventually) positive terms. This makes the following result\footnote{This is easily proved using the Squeeze Theroem} particularly important

\theorem If $|a_n|\to 0$ as $n\to\infty$, then $a_n\to 0$ as $n\to\infty$.

Geometric sequences $a_n = r^n$ (increasing powers of some constant $r$) will also play an important role in what is to follow. The following theorem is proved in section 11.1 of the text.

\theorem[Geometric Sequences] The sequence $a_n = r^n$ is convergent if $-1 < r\leq 1$ and divergent for all other values of $r$. For those values of $r$ for which the sequence converges, we have
  \begin{equation*}
    \lim_{n\to\infty} r^n = \begin{cases}
      0 &\text{if}\quad -1 < r < 1\\
      1 &\text{if}\quad r = 1
  \end{cases}
  \end{equation*}

\theoremstyle{definition}
\begin{definition}[Increasing Sequence]
  A sequence $a_n$ is \emph{increasing} if $a_n < a_{n+1}$
\end{definition}
\begin{definition}[Decreasing Sequence]
  A sequence $a_n$ is \emph{decreasing} if $a_n > a_{n+1}$
\end{definition}
\begin{definition}[Monotone Sequence]
  A sequence $a_n$ is \emph{monotone} if it is either increasing or decreasing.
\end{definition}
\begin{definition}[Bounded Sequence]
  A sequence $a_n$ is \emph{bounded above} if there exists a number $M$ such that $a_n \leq M$, for all $n$. It is \emph{bounded below} if there exists a number $m$ such that $m \leq a_n$, for all $n$. It is \emph{bounded} if it is both bounded above and bounded below.
\end{definition}
Finally, an important theorem is
\theorem[Monotone Sequence Theorem] Every bounded (eventually) monotonic sequence is convergent.
\newpage
\subsection*{Problems}
Determine if the following sequences converge or diverge. If the sequence converges, give its limiting value. If the sequence diverges to $\pm\infty$, state so. If the sequence diverges but not to $\pm\infty$, just say that it diverges.
\begin{enumerate}
  \item $a_n = \dfrac{2n^2}{3n^2-50n+8000}$
  \item $a_n = \dfrac{30n^2+12n-6}{3n^3}$
  \item $a_n = \dfrac{5n^4+n^2+3}{n^3-n+1}$
  \item $a_n = 3^n\,7^{-n}$
  \item $a_n = \dfrac{2^n - 5^n}{4^n}$
  \item $a_n = e^{2n/(n+2)}$
  \item $a_n = \cos\left(\dfrac{n\pi}{n+1}\right)$
  \item $a_n = \ln(n+1) - \ln(n)$ Hint: Use a property of logarithms.
\end{enumerate}
\end{document}
