\documentclass[12pt,letterpaper,fleqn]{article}

%       amslatex provides nice math extensions for typesetting mathematics
\usepackage{amsmath}
\usepackage{amsfonts}
\usepackage{tmmaths}
\usepackage{sympytex}

%       pstricks provides powerful environments for incorporating postscript into a
%       TeX/LaTeX document. You must have a postscript printer and a package like
%       dvips to convert the DVI file to a PS file.
%\usepackage{pst-all}
%\usepackage{pstricks,pst-plot}
%\usepackage{pst-coil,pst-node}

%  This package provides native tex support for numbered grids. The syntax is:
%  \graphpaper[spc](x_lowleft,y_lowleft)(x_upperright,y_upperright)

%\usepackage{graphpap}
%\usepackage{float}

%  The package below must be initialized with "\initfloatingfigs" immediately after the
%  "\begin{document} command.
%\usepackage{floatfig}

\usepackage{graphicx}
\graphicspath{{i:/mytex/graphics}}
\DeclareGraphicsExtensions{.ps,.eps}

%       tst is a package for the creation of exams, quizzes and tests. the include
%       file mathstuf (see below) provides many abbreviations for these environments.
%\usepackage{tst}

%       epsfig is a package which provides for the inclusion of Encapsulated PostScript
%       files in a document.
%\usepackage{epsfig}
%\usepackage{epic,eepic}
\include{mathstuf}
\usepackage[total={7.25in,10in},top=0.25in,left=0.75in,includehead]{geometry}
\usepackage{fancyhdr}
\pagestyle{fancy}
\lhead{Math 252}
\rhead{\large Name\makebox[2in]{\hrulefill}}
\chead{\LARGE Exploration 13}
%\lfoot{\today}
\cfoot{}
%\rfoot{\thepage}
\renewcommand{\headrulewidth}{0.4pt}
\renewcommand{\footrulewidth}{0.4pt}
\setlength{\parindent}{0pt}
\setlength{\parskip}{2ex}

\newcounter{tf}[enumi]
\newenvironment{tf}[0]{\begin{list}%
{\alph{tf}. \makebox[5em]{True\hfill False}}%
{\usecounter{tf}\setlength{\labelwidth}{7em}%
\setlength{\leftmargin}{3.5cm}%
\setlength{\labelsep}{1cm}}}%
{\end{list}}

%\usepackage{epic,eepic}
\newcommand{\numline}{%
%\newcounter{mark}%
%\setcounter{mark}{-1}%
\setlength{\unitlength}{0.1in}%
\begin{picture}(0,0)%
\thicklines%
\put(0,0){\line(1,0){60}}%
\multiput(0,0)(10,0){7}{\line(0,-1){1}%
\makebox(0,-1.5)[t]{\arabic{mark}}\stepcounter{mark}}%
%
\thinlines%
\multiput(0,0)(5,0){12}{\line(0,-1){0.5}}%
\multiput(0,0)(1,0){60}{\line(0,-1){0.3}}%
%\put(-5,265){\makebox(0,0)[l]{{\bf cm}}}%
\end{picture}}%

\newcommand{\ds}{\displaystyle}
\usepackage{amsfonts}


\let\oldhat\hat
\renewcommand{\hat}[1]{\oldhat{\boldsymbol{\mathbf{#1}}}}
\newcommand{\lv}[1]{\ensuremath{\langle #1 \rangle}}
\renewcommand{\i}{\ensuremath{\hat{\imath}}}
\renewcommand{\j}{\ensuremath{\hat{\jmath}}}
\renewcommand{\k}{\ensuremath{\mathbf{\oldhat{k}}}}
\newcommand{\ora}[1]{\ensuremath{\overrightarrow{#1}}}
\renewcommand{\vec}[1]{\ensuremath{\mathbf{#1}}}
\renewcommand{\v}[1]{\ensuremath{\vec{#1}}}
\newcommand{\abs}[1]{\ensuremath{\lvert #1 \rvert}}

\usepackage{tabularx}
\usepackage{paralist}
\newcommand{\red}[1]{\textcolor{red}{#1}}
\newcommand{\blue}[1]{\textcolor{blue}{#1}}
% \newcommand{\ans}[1]{\quad\fbox{answer: \red{#1}}}
\newcommand{\ans}[1]{\mbox{{\bf Ans:} \blue{#1}}}
\newcommand{\dd}[2][]{\ensuremath{\frac{\text{d}#1}{\text{d}#2}}}
\newcommand{\eval}[2]{\ensuremath{\left.#1\right|_{#2}}}

\usepackage{amsthm}

\theoremstyle{definition}
\newtheorem*{definition}{Definition}

\usepackage{enumitem}
\usepackage{subfig}

\begin{document}
\section*{Trigonometric Integrals}
In this exploration we consider integrals of trigonometric functions. First make sure you recall the \emph{derivatives} of all the trigonometric functions.
\begin{enumerate}
  \item Write down the derivatives of: $\sin x$, $\cos x$, $\tan x$, $\cot x$, $\sec x$, $\csc x$.
\end{enumerate}

To deal with trigonometric integrals you will need to know a few trigonometric identities\footnote{A more complete list is available in the modules section of Canvas}. Beyond the basic identities, you will need to know:
\begin{itemize}
  \item $\sin^2 x + \cos^2 x = 1$
  \item $\tan^2 x + 1 = \sec^2 x$
  \item $1 + \cot^2 x = \csc^2 x$
  \item $\sin 2x = 2\sin x\cos x$
  \item $\cos 2x = \cos^2 x - \sin^2 x = 1 - 2\sin^2 x = 2\cos^2 x - 1$
  \item $\sin^2 x = \frac{1}{2}(1 - \cos 2x)$
  \item $\cos^2 x = \frac{1}{2}(1 + \cos 2x)$
\end{itemize}
Additionally, you will need to know the following properties of logarithms:
\begin{itemize}
  \item $\ln(xy) = \ln(x) + \ln(y)$
  \item $\ln(x/y) = \ln(x) - \ln(y)$
  \item $\ln(x^r) = r\ln(x)$
\end{itemize}

\subsection*{Integrals of all the trigonometric functions}
Evaluate each of the following integrals:
\begin{enumerate}[resume]
  \item $\int\sin x\;dx$
  \item $\int\cos x\;dx$
  \item $\int\tan x\;dx$ Hint: $\tan x = \dfrac{\sin x}{\cos x}$ and use substitution. Use properties of logarithms to write your answer in terms of $\sec x$.
  \item $\int\cot x\;dx$ Hint: $\cot x = \dfrac{\cos x}{\sin x}$ and use substitution. Use properties of logarithms to write your answer in terms of $\csc x$.
  \item $\int\sec x\;dx$ Hint: Multiply the integrand by $\dfrac{\sec x + \tan x}{\sec x + \tan x}$ and use substitution.
  \item $\int\csc x\;dx$ Hint: Multiply the integrand by $\dfrac{\csc x + \cot x}{\csc x + \cot x}$ and use substitution.
\end{enumerate}
% \subsection*{$\pmb{\int\sin^m x \cos^n x\;dx}$, with $\pmb{n, m}$ being non-negative integers}
%   To evaluate integrals of this type you will need to use some trigonometric identities to rewrite the integrand.
%
%   Evaluate the following integrals
%   \begin{enumerate}[resume]
%     \item $\int\cos x\sin^8 x\;dx$
%     \item $\int\cos^3 x\;dx$
%     \item $\int\sin^5 x\;dx$
%     \item $\int\sin^2 x\;dx$
%     \item $\int\cos^4 x\;dx$
%   \end{enumerate}
\end{document}
