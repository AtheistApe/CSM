\documentclass[12pt,letterpaper,fleqn]{article}

%       amslatex provides nice math extensions for typesetting mathematics
\usepackage{amsmath}
\usepackage{amsfonts}
\usepackage{tmmaths}
\usepackage{sympytex}

%       pstricks provides powerful environments for incorporating postscript into a
%       TeX/LaTeX document. You must have a postscript printer and a package like
%       dvips to convert the DVI file to a PS file.
%\usepackage{pst-all}
%\usepackage{pstricks,pst-plot}
%\usepackage{pst-coil,pst-node}

%  This package provides native tex support for numbered grids. The syntax is:
%  \graphpaper[spc](x_lowleft,y_lowleft)(x_upperright,y_upperright)

%\usepackage{graphpap}
%\usepackage{float}

%  The package below must be initialized with "\initfloatingfigs" immediately after the
%  "\begin{document} command.
%\usepackage{floatfig}

\usepackage{graphicx}
\graphicspath{{i:/mytex/graphics}}
\DeclareGraphicsExtensions{.ps,.eps}

%       tst is a package for the creation of exams, quizzes and tests. the include
%       file mathstuf (see below) provides many abbreviations for these environments.
%\usepackage{tst}

%       epsfig is a package which provides for the inclusion of Encapsulated PostScript
%       files in a document.
%\usepackage{epsfig}
%\usepackage{epic,eepic}
\include{mathstuf}
\usepackage[total={7.25in,10in},top=0.25in,left=0.75in,includehead]{geometry}
\usepackage{fancyhdr}
\pagestyle{fancy}
\lhead{Math 252}
\rhead{\large Name\makebox[2in]{\hrulefill}}
\chead{\LARGE Exploration 11}
%\lfoot{\today}
\cfoot{}
%\rfoot{\thepage}
\renewcommand{\headrulewidth}{0.4pt}
\renewcommand{\footrulewidth}{0.4pt}
\setlength{\parindent}{0pt}
\setlength{\parskip}{2ex}

\newcounter{tf}[enumi]
\newenvironment{tf}[0]{\begin{list}%
{\alph{tf}. \makebox[5em]{True\hfill False}}%
{\usecounter{tf}\setlength{\labelwidth}{7em}%
\setlength{\leftmargin}{3.5cm}%
\setlength{\labelsep}{1cm}}}%
{\end{list}}

%\usepackage{epic,eepic}
\newcommand{\numline}{%
%\newcounter{mark}%
%\setcounter{mark}{-1}%
\setlength{\unitlength}{0.1in}%
\begin{picture}(0,0)%
\thicklines%
\put(0,0){\line(1,0){60}}%
\multiput(0,0)(10,0){7}{\line(0,-1){1}%
\makebox(0,-1.5)[t]{\arabic{mark}}\stepcounter{mark}}%
%
\thinlines%
\multiput(0,0)(5,0){12}{\line(0,-1){0.5}}%
\multiput(0,0)(1,0){60}{\line(0,-1){0.3}}%
%\put(-5,265){\makebox(0,0)[l]{{\bf cm}}}%
\end{picture}}%

\newcommand{\ds}{\displaystyle}
\usepackage{amsfonts}


\let\oldhat\hat
\renewcommand{\hat}[1]{\oldhat{\boldsymbol{\mathbf{#1}}}}
\newcommand{\lv}[1]{\ensuremath{\langle #1 \rangle}}
\renewcommand{\i}{\ensuremath{\hat{\imath}}}
\renewcommand{\j}{\ensuremath{\hat{\jmath}}}
\renewcommand{\k}{\ensuremath{\mathbf{\oldhat{k}}}}
\newcommand{\ora}[1]{\ensuremath{\overrightarrow{#1}}}
\renewcommand{\vec}[1]{\ensuremath{\mathbf{#1}}}
\renewcommand{\v}[1]{\ensuremath{\vec{#1}}}
\newcommand{\abs}[1]{\ensuremath{\lvert #1 \rvert}}

\usepackage{tabularx}
\usepackage{paralist}
\newcommand{\red}[1]{\textcolor{red}{#1}}
\newcommand{\blue}[1]{\textcolor{blue}{#1}}
% \newcommand{\ans}[1]{\quad\fbox{answer: \red{#1}}}
\newcommand{\ans}[1]{\mbox{{\bf Ans:} \blue{#1}}}
\newcommand{\dd}[2][]{\ensuremath{\frac{\text{d}#1}{\text{d}#2}}}
\newcommand{\eval}[2]{\ensuremath{\left.#1\right|_{#2}}}

\usepackage{amsthm}

\theoremstyle{definition}
\newtheorem*{definition}{Definition}

\usepackage{enumitem}
\usepackage{subfig}

\begin{document}
\section*{Integration by Parts}
In this exploration we will evaluate integrals using the method of ``Integration by Parts''.

Recall that an antiderivative of a function $f$ is a function $F$ such that $F' = f$. For example, an antiderivative of $\cos(x)$ is a function whose derivative \emph{is} $\cos(x)$. Therefore the functions $\sin(x)$, $\sin(x) + 3$, $\sin(x) - 1$ and, in general, $\sin(x) + C$ for any constant $C$ are all antiderivatives of $\cos(x)$.

We represent the antiderivatives of a function $f$ as $\int f$. Therefore, we can express the defintion of the antiderivative of a function $f$ in symbolic form as
\begin{equation*}
  \int f = F \iff F' = f
\end{equation*}
Suppose now that $f$ and $g$ are two functions. The product rule for differentiation allows us to express the derivative of the product $fg$ of the functions as a combination of the derivatives of the functions $f$ and $g$. In particular
\begin{equation*}
  (fg)' = f'g + fg'
\end{equation*}
Hence
\begin{equation*}
  \int(f'g + fg') = fg
\end{equation*}
since $fg$ is a function whose derivative is $f'g + fg'$.

Working with this integral expression, we get
\begin{equation*}
  \int(f'g + fg') = fg \implies \int f'g + \int fg' = fg
\end{equation*}
Solving this last equation for $\int fg'$ we have the integration by parts formula
\begin{equation*}
  \int fg' = fg - \int f'g
\end{equation*}
This formula allows us to trade in one integral, which may be difficult, with another integral, which may be simpler.
\subsection*{Examples}
\begin{enumerate}
  \item Consider $\int 2x \cos x\;dx = \int 2x (\sin x)'\;dx$. Using the integration by parts formula we have
  \begin{align*}
    \int 2x (\sin x)'\;dx &= 2x \sin x - \int (2x)' \sin x\;dx\\
    &= 2x \sin x - \int 2\sin x\;dx\\
    &= 2x \sin x + 2\cos x + C
  \end{align*}
  \item Consider $\int \ln x\;dx = \int \ln x (x)'\;dx$. Using the integration by parts formula we have
  \begin{align*}
    \int \ln x (x)'\;dx &= x\ln x - \int (\ln x)' x\;dx\\
    &= x\ln x - \int\frac{1}{x} \cdot x\;dx\\
    &= x\ln x - \int 1\;dx\\
    &= x\ln x - x + C
  \end{align*}
\end{enumerate}
\subsection*{Problems}
Using the integration by parts formula, evaluate each integral.
\begin{enumerate}
  \item $\int 4x e^x\;dx$
  \item $\int x \sin x\;dx$
  \item $\int 2x \cos 3x\;dx$
  \item $\int x^2 e^x\;dx$
  \item $\int x \sec^2 x\;dx$ Hint: $(\tan x)' = \sec^2 x$.
\end{enumerate}

\end{document}
