\documentclass[12pt,letterpaper,fleqn]{article}

%       amslatex provides nice math extensions for typesetting mathematics
\usepackage{amsmath}
\usepackage{amsfonts}
\usepackage{tmmaths}
\usepackage{sympytex}

%       pstricks provides powerful environments for incorporating postscript into a
%       TeX/LaTeX document. You must have a postscript printer and a package like
%       dvips to convert the DVI file to a PS file.
%\usepackage{pst-all}
%\usepackage{pstricks,pst-plot}
%\usepackage{pst-coil,pst-node}

%  This package provides native tex support for numbered grids. The syntax is:
%  \graphpaper[spc](x_lowleft,y_lowleft)(x_upperright,y_upperright)

%\usepackage{graphpap}
%\usepackage{float}

%  The package below must be initialized with "\initfloatingfigs" immediately after the
%  "\begin{document} command.
%\usepackage{floatfig}

\usepackage{graphicx}
\graphicspath{{i:/mytex/graphics}}
\DeclareGraphicsExtensions{.ps,.eps}

%       tst is a package for the creation of exams, quizzes and tests. the include
%       file mathstuf (see below) provides many abbreviations for these environments.
%\usepackage{tst}

%       epsfig is a package which provides for the inclusion of Encapsulated PostScript
%       files in a document.
%\usepackage{epsfig}
%\usepackage{epic,eepic}
\include{mathstuf}
\usepackage[total={7.25in,10in},top=0.25in,left=0.75in,includehead]{geometry}
\usepackage{fancyhdr}
\pagestyle{fancy}
\lhead{Math 252}
\rhead{\large Name\makebox[2in]{\hrulefill}}
\chead{\LARGE Exploration 17}
%\lfoot{\today}
\cfoot{}
%\rfoot{\thepage}
\renewcommand{\headrulewidth}{0.4pt}
\renewcommand{\footrulewidth}{0.4pt}
\setlength{\parindent}{0pt}
\setlength{\parskip}{2ex}

\newcounter{tf}[enumi]
\newenvironment{tf}[0]{\begin{list}%
{\alph{tf}. \makebox[5em]{True\hfill False}}%
{\usecounter{tf}\setlength{\labelwidth}{7em}%
\setlength{\leftmargin}{3.5cm}%
\setlength{\labelsep}{1cm}}}%
{\end{list}}

%\usepackage{epic,eepic}
\newcommand{\numline}{%
%\newcounter{mark}%
%\setcounter{mark}{-1}%
\setlength{\unitlength}{0.1in}%
\begin{picture}(0,0)%
\thicklines%
\put(0,0){\line(1,0){60}}%
\multiput(0,0)(10,0){7}{\line(0,-1){1}%
\makebox(0,-1.5)[t]{\arabic{mark}}\stepcounter{mark}}%
%
\thinlines%
\multiput(0,0)(5,0){12}{\line(0,-1){0.5}}%
\multiput(0,0)(1,0){60}{\line(0,-1){0.3}}%
%\put(-5,265){\makebox(0,0)[l]{{\bf cm}}}%
\end{picture}}%

\newcommand{\ds}{\displaystyle}
\usepackage{amsfonts}


\let\oldhat\hat
\renewcommand{\hat}[1]{\oldhat{\boldsymbol{\mathbf{#1}}}}
\newcommand{\lv}[1]{\ensuremath{\langle #1 \rangle}}
\renewcommand{\i}{\ensuremath{\hat{\imath}}}
\renewcommand{\j}{\ensuremath{\hat{\jmath}}}
\renewcommand{\k}{\ensuremath{\mathbf{\oldhat{k}}}}
\newcommand{\ora}[1]{\ensuremath{\overrightarrow{#1}}}
\renewcommand{\vec}[1]{\ensuremath{\mathbf{#1}}}
\renewcommand{\v}[1]{\ensuremath{\vec{#1}}}
\newcommand{\abs}[1]{\ensuremath{\lvert #1 \rvert}}

\usepackage{tabularx}
\usepackage{paralist}
\newcommand{\red}[1]{\textcolor{red}{#1}}
\newcommand{\blue}[1]{\textcolor{blue}{#1}}
% \newcommand{\ans}[1]{\quad\fbox{answer: \red{#1}}}
\newcommand{\ans}[1]{\mbox{{\bf Ans:} \blue{#1}}}
\newcommand{\dd}[2][]{\ensuremath{\frac{\text{d}#1}{\text{d}#2}}}
\newcommand{\eval}[2]{\ensuremath{\left.#1\right|_{#2}}}

\usepackage{amsthm}

\theoremstyle{definition}
\newtheorem*{definition}{Definition}

\usepackage{enumitem}
\usepackage{subfig}

\newcommand{\dint}{\ensuremath{\displaystyle\int}}

\begin{document}
\section*{Integration of Rational Functions, Part 1}
  A rational function $f(x)$ is a ratio of polynomial functions.
\begin{equation*}
  f(x) = \frac{p(x)}{q(x)}
\end{equation*}
where $p$ and $q$ are polynomials.

The rational function $f$ is called ``proper'' if $\text{deg}(p) < \text{deg}(q)$. It is called ``improper'' if $\text{deg}(p) \geq \text{deg}(q)$. Any improper rational function can be expressed as the sum of a polynomial part and a proper rational function part using polynomial division. Since we know how to integrate polynomials, the remaining problem is how to integrate proper rational functions.

It turns out that it is (in principle) \emph{always} possible to integrate \emph{any} proper rational function using the method of ``partial fraction decomposition''. The idea is to write (decompose) the proper fraction (rational function) as a sum of ``partial fractions'' each of which can be integrated by standard methods. A theorem from Linear Algebra proves that this is always possible.

These partial fractions are built from the linear and irreducible quadratic factors of the denominator $q(x)$ in the following way:\footnote{ The Fundamental Theorem of Algebra proves that $q$ can be factored as a product of linear and irreducible quadratic factors.}
\begin{itemize}
  \item Factors of $q$ of the form $(ax + b)^n$ admit terms $\frac{A_1}{ax + b} + \frac{A_2}{(ax + b)^2} + \cdots + \frac{A_n}{(ax + b)^n}$.
  \item Factors of $q$ of the form $(ax^2 + bx + c)^m$ admit terms $\frac{B_1x + C_1}{ax^2 + bx + c} + \frac{B_2x + C_2}{(ax^2 + bx + c)^2} + \cdots + \frac{B_mx + C_m}{(ax^2 + bx + c)^m}$, where $ax^2 + bx + c$ is an irreducible quadratic polynomial.
\end{itemize}
where $m$ and $n$ are positive integers and the $A$'s, $B$'s, and $C$'s are constants.

For example,
\begin{equation*}
  \frac{x^2 - x + 3}{(x+1)(2x-1)^3(x^2+1)^2} = \frac{A_1}{x+1} + \frac{B_1}{2x-1} + \frac{B_2}{(2x-1)^2} + \frac{B_3}{(2x-1)^3} + \frac{C_1x+D_1}{x^2+1} + \frac{C_2x+D_2}{(x^2+1)^2}
\end{equation*}
where the $A$'s, $B$'s, $C$'s and $D$'s are constants to be determined.
\newpage
\subsection*{Problems}
Find the partial fraction decomposition of each of the following proper rational functions as in the above example.
\begin{enumerate}
  \item $\dfrac{3}{(x + 1)(x - 2)}$
  \item $\dfrac{2x-1}{x^2(x^2 + x + 1)}$
  \item $\dfrac{3x^2 - 4}{(2x-1)^2 (x+3) (x-1)^3}$
  \item $\dfrac{x - 1}{x^2 - 9}$
  \item $\dfrac{2x^2 + 3}{x^3 + x^2 - 6x}$
  \item $\dfrac{x^3}{x^4 - 1}$
  \item $\dfrac{2x + 5}{3x^4 + 2x^2}$
  \item $\dfrac{x^2}{x^3 + x^2 - 4x - 4}$
\end{enumerate}
\end{document}
