\documentclass[12pt,letterpaper,fleqn]{article}

%       amslatex provides nice math extensions for typesetting mathematics
\usepackage{amsmath}
\usepackage{amsfonts}
\usepackage{tmmaths}
\usepackage{sympytex}

%       pstricks provides powerful environments for incorporating postscript into a
%       TeX/LaTeX document. You must have a postscript printer and a package like
%       dvips to convert the DVI file to a PS file.
%\usepackage{pst-all}
%\usepackage{pstricks,pst-plot}
%\usepackage{pst-coil,pst-node}

%  This package provides native tex support for numbered grids. The syntax is:
%  \graphpaper[spc](x_lowleft,y_lowleft)(x_upperright,y_upperright)

%\usepackage{graphpap}
%\usepackage{float}

%  The package below must be initialized with "\initfloatingfigs" immediately after the
%  "\begin{document} command.
%\usepackage{floatfig}

\usepackage{graphicx}
\graphicspath{{i:/mytex/graphics}}
\DeclareGraphicsExtensions{.ps,.eps}

%       tst is a package for the creation of exams, quizzes and tests. the include
%       file mathstuf (see below) provides many abbreviations for these environments.
%\usepackage{tst}

%       epsfig is a package which provides for the inclusion of Encapsulated PostScript
%       files in a document.
%\usepackage{epsfig}
%\usepackage{epic,eepic}
\include{mathstuf}
\usepackage[total={7.25in,10in},top=0.25in,left=0.75in,includehead]{geometry}
\usepackage{fancyhdr}
\pagestyle{fancy}
\lhead{Math 252}
\rhead{\large Name\makebox[2in]{\hrulefill}}
\chead{\LARGE Exploration 18}
%\lfoot{\today}
\cfoot{}
%\rfoot{\thepage}
\renewcommand{\headrulewidth}{0.4pt}
\renewcommand{\footrulewidth}{0.4pt}
\setlength{\parindent}{0pt}
\setlength{\parskip}{2ex}

\newcounter{tf}[enumi]
\newenvironment{tf}[0]{\begin{list}%
{\alph{tf}. \makebox[5em]{True\hfill False}}%
{\usecounter{tf}\setlength{\labelwidth}{7em}%
\setlength{\leftmargin}{3.5cm}%
\setlength{\labelsep}{1cm}}}%
{\end{list}}

%\usepackage{epic,eepic}
\newcommand{\numline}{%
%\newcounter{mark}%
%\setcounter{mark}{-1}%
\setlength{\unitlength}{0.1in}%
\begin{picture}(0,0)%
\thicklines%
\put(0,0){\line(1,0){60}}%
\multiput(0,0)(10,0){7}{\line(0,-1){1}%
\makebox(0,-1.5)[t]{\arabic{mark}}\stepcounter{mark}}%
%
\thinlines%
\multiput(0,0)(5,0){12}{\line(0,-1){0.5}}%
\multiput(0,0)(1,0){60}{\line(0,-1){0.3}}%
%\put(-5,265){\makebox(0,0)[l]{{\bf cm}}}%
\end{picture}}%

\newcommand{\ds}{\displaystyle}
\usepackage{amsfonts}


\let\oldhat\hat
\renewcommand{\hat}[1]{\oldhat{\boldsymbol{\mathbf{#1}}}}
\newcommand{\lv}[1]{\ensuremath{\langle #1 \rangle}}
\renewcommand{\i}{\ensuremath{\hat{\imath}}}
\renewcommand{\j}{\ensuremath{\hat{\jmath}}}
\renewcommand{\k}{\ensuremath{\mathbf{\oldhat{k}}}}
\newcommand{\ora}[1]{\ensuremath{\overrightarrow{#1}}}
\renewcommand{\vec}[1]{\ensuremath{\mathbf{#1}}}
\renewcommand{\v}[1]{\ensuremath{\vec{#1}}}
\newcommand{\abs}[1]{\ensuremath{\lvert #1 \rvert}}

\usepackage{tabularx}
\usepackage{paralist}
\newcommand{\red}[1]{\textcolor{red}{#1}}
\newcommand{\blue}[1]{\textcolor{blue}{#1}}
% \newcommand{\ans}[1]{\quad\fbox{answer: \red{#1}}}
\newcommand{\ans}[1]{\mbox{{\bf Ans:} \blue{#1}}}
\newcommand{\dd}[2][]{\ensuremath{\frac{\text{d}#1}{\text{d}#2}}}
\newcommand{\eval}[2]{\ensuremath{\left.#1\right|_{#2}}}

\usepackage{amsthm}

\theoremstyle{definition}
\newtheorem*{definition}{Definition}

\usepackage{enumitem}
\usepackage{subfig}

\newcommand{\dint}{\ensuremath{\displaystyle\int}}

\begin{document}
\section*{Integration of Rational Functions, Part 2}
We saw in exploration 17 that any proper rational function $p(x)/q(x)$ can be expressed as a sum of ``partial fractions'' where the denominators of those fractions are built from the linear and irreducible quadratic factors of $q(x)$. In writing the partial fraction decomposition of $p(x)/q(x)$, we are left with a large number of undetermined constants that must be found. In this exploration we will learn how to find the values of these constants.

Suppose we have decomposed the proper rational function $p(x)/q(x)$ into the sum of its partial fractions
\begin{equation*}
 \frac{p(x)}{q(x)} = (\text{sum of partial fractions})
\end{equation*}
Multiplying both sides of this equation by $q(x)$ gives
\begin{equation*}
 p(x) = (\text{sum of partial fractions})\,q(x)
\end{equation*}
Because of the nature of the partial fraction decomposition of $p(x)/q(x)$, the expression on the right hand side of this equation is a polynomial $r(x)$. The polynomial $p(x)$ equals the polynomial $r(x)$ if and only if all the coefficients of the different powers of $x$ in $p(x)$ match the coefficients of the corresponding powers of $x$ in $r(x)$.

By equating the coefficients of corresponding powers in the polynomials $p(x)$ and $r(x)$, we get a system of linear equations in the undetermined constants.\footnote{The partial fraction decomposition of $p(x)/q(x)$ guarantees that there will be the same number of linearly independent equations as there are undetermined constants, thereby insuring a unique solution of the system.} Solving this system determines the values of these constants, thereby determining the partial fraction decomposition of $p(x)/q(x)$.
\subsection*{Example}
Consider the partial fraction decomposition of the proper rational function
\begin{equation*}
 \frac{x^2}{(x-1)(x^2 + 1)} = \frac{A}{x-1} + \frac{Bx + C}{x^2 + 1}
\end{equation*}
where $A$, $B$, and $C$ are the undetermined constants.

Multiplying through by $(x-1)(x^2 + 1)$ gives
\begin{align*}
 x^2 & = \left(\frac{A}{x-1} + \frac{Bx + C}{x^2 + 1}\right)(x-1)(x^2 + 1) \\[1ex]
     & = A(x^2 + 1) + (Bx + C)(x-1)                                        \\[1ex]
     & = Ax^2 + A + Bx^2 - Bx + Cx - C                                     \\[1ex]
     & = (A + B)x^2 + (C - B)x + (A - C)
\end{align*}
Equating the coefficients of $x^2$, $x$, and the constants on both sides of the equation gives us a system of three linear equations in the three undetermined constants $A$, $B$, and $C$
\begin{align}
 A + B & = 1                \\
 C - B & = 0 \implies C = B \\
 A - C & = 0 \implies A = C
\end{align}
From equations (2) and (3) we see that $A = B = C$. Therefore, equation (1) implies that $2A = 1$, hence $A = B = C = 1/2$.\footnote{There are methods in linear algebra using matrices that makes solving a system of $n$ equations in $n$ unknowns a simple mechanical procedure.} Thus, we have determined that
\begin{equation*}
 \frac{x^2}{(x-1)(x^2 + 1)} = \frac{1}{2}\cdot\frac{1}{x-1} + \frac{1}{2}\cdot\frac{x + 1}{x^2 + 1}
\end{equation*}
\subsection*{Problem}
\begin{enumerate}
 \item Evaluate $\displaystyle\int\frac{x^2}{(x-1)(x^2 + 1)}\;dx$
\end{enumerate}
\subsection*{Problems}
Find the partial fraction decomposition of each of the following proper rational functions. Solve a system of linear equations to find the values of the undetermined constants as in the previous example.
\begin{enumerate}[resume]
 \item $\dfrac{x - 1}{x^2 - 9}$
 \item $\dfrac{x - 7}{x^2 - 2x - 3}$
 \item $\dfrac{6x + 2}{4x^2 + 4x + 1}$
 \item $\dfrac{3x^3 - 8}{x^4 + 4x^2}$
\end{enumerate}
\subsection*{Problems}
Integrate each of the rational functions in problems (2)--(5).

\end{document}
