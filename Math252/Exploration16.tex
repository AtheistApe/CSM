\documentclass[12pt,letterpaper,fleqn]{article}

%       amslatex provides nice math extensions for typesetting mathematics
\usepackage{amsmath}
\usepackage{amsfonts}
\usepackage{tmmaths}
\usepackage{sympytex}

%       pstricks provides powerful environments for incorporating postscript into a
%       TeX/LaTeX document. You must have a postscript printer and a package like
%       dvips to convert the DVI file to a PS file.
%\usepackage{pst-all}
%\usepackage{pstricks,pst-plot}
%\usepackage{pst-coil,pst-node}

%  This package provides native tex support for numbered grids. The syntax is:
%  \graphpaper[spc](x_lowleft,y_lowleft)(x_upperright,y_upperright)

%\usepackage{graphpap}
%\usepackage{float}

%  The package below must be initialized with "\initfloatingfigs" immediately after the
%  "\begin{document} command.
%\usepackage{floatfig}

\usepackage{graphicx}
\graphicspath{{i:/mytex/graphics}}
\DeclareGraphicsExtensions{.ps,.eps}

%       tst is a package for the creation of exams, quizzes and tests. the include
%       file mathstuf (see below) provides many abbreviations for these environments.
%\usepackage{tst}

%       epsfig is a package which provides for the inclusion of Encapsulated PostScript
%       files in a document.
%\usepackage{epsfig}
%\usepackage{epic,eepic}
\include{mathstuf}
\usepackage[total={7.25in,10in},top=0.25in,left=0.75in,includehead]{geometry}
\usepackage{fancyhdr}
\pagestyle{fancy}
\lhead{Math 252}
\rhead{\large Name\makebox[2in]{\hrulefill}}
\chead{\LARGE Exploration 16}
%\lfoot{\today}
\cfoot{}
%\rfoot{\thepage}
\renewcommand{\headrulewidth}{0.4pt}
\renewcommand{\footrulewidth}{0.4pt}
\setlength{\parindent}{0pt}
\setlength{\parskip}{2ex}

\newcounter{tf}[enumi]
\newenvironment{tf}[0]{\begin{list}%
{\alph{tf}. \makebox[5em]{True\hfill False}}%
{\usecounter{tf}\setlength{\labelwidth}{7em}%
\setlength{\leftmargin}{3.5cm}%
\setlength{\labelsep}{1cm}}}%
{\end{list}}

%\usepackage{epic,eepic}
\newcommand{\numline}{%
%\newcounter{mark}%
%\setcounter{mark}{-1}%
\setlength{\unitlength}{0.1in}%
\begin{picture}(0,0)%
\thicklines%
\put(0,0){\line(1,0){60}}%
\multiput(0,0)(10,0){7}{\line(0,-1){1}%
\makebox(0,-1.5)[t]{\arabic{mark}}\stepcounter{mark}}%
%
\thinlines%
\multiput(0,0)(5,0){12}{\line(0,-1){0.5}}%
\multiput(0,0)(1,0){60}{\line(0,-1){0.3}}%
%\put(-5,265){\makebox(0,0)[l]{{\bf cm}}}%
\end{picture}}%

\newcommand{\ds}{\displaystyle}
\usepackage{amsfonts}


\let\oldhat\hat
\renewcommand{\hat}[1]{\oldhat{\boldsymbol{\mathbf{#1}}}}
\newcommand{\lv}[1]{\ensuremath{\langle #1 \rangle}}
\renewcommand{\i}{\ensuremath{\hat{\imath}}}
\renewcommand{\j}{\ensuremath{\hat{\jmath}}}
\renewcommand{\k}{\ensuremath{\mathbf{\oldhat{k}}}}
\newcommand{\ora}[1]{\ensuremath{\overrightarrow{#1}}}
\renewcommand{\vec}[1]{\ensuremath{\mathbf{#1}}}
\renewcommand{\v}[1]{\ensuremath{\vec{#1}}}
\newcommand{\abs}[1]{\ensuremath{\lvert #1 \rvert}}

\usepackage{tabularx}
\usepackage{paralist}
\newcommand{\red}[1]{\textcolor{red}{#1}}
\newcommand{\blue}[1]{\textcolor{blue}{#1}}
% \newcommand{\ans}[1]{\quad\fbox{answer: \red{#1}}}
\newcommand{\ans}[1]{\mbox{{\bf Ans:} \blue{#1}}}
\newcommand{\dd}[2][]{\ensuremath{\frac{\text{d}#1}{\text{d}#2}}}
\newcommand{\eval}[2]{\ensuremath{\left.#1\right|_{#2}}}

\usepackage{amsthm}

\theoremstyle{definition}
\newtheorem*{definition}{Definition}

\usepackage{enumitem}
\usepackage{subfig}

\newcommand{\dint}{\ensuremath{\displaystyle\int}}

\begin{document}
\section*{Review for Quiz on Integration by Parts and Trig.\ Integrals}
Each of the integrals below can be resolved using either (or a combination of) integration by parts, substitution, trig.\ identities or the special cases described in exploration 14.\\[1.5ex] For the quiz you should know (memorize) the integrals of the six trig.\ functions and the following trig.\ identities:
\begin{itemize}
  \item $\sin^2 x + \cos^2 x = 1$
  \item $\tan^2 x + 1 = \sec^2 x$
  \item $\sin(2x) = 2\sin x \cos x$
  \item $\cos^2 x = \frac{1}{2}\left(1 + \cos 2x\right)$
  \item $\sin^2 x = \frac{1}{2}\left(1 - \cos 2x\right)$
\end{itemize}
\subsection*{Problems}
Evaluate each of the following integrals without using a calculator. You \emph{must} show your work.
\begin{enumerate}
  \item $\dint 3x e^{2x}\;dx$
  \item $\dint \sin^5 y\cos y\;dy$
  \item $\dint_1^e 2u \ln u\;du$
  \item $\dint_0^{\pi/4}\tan^2 \theta\;d\theta$
  \item $\dint 2x\sec 3x\tan 3x\;dx$
  \item $\dint \sin^2 x\;dx$
  \item $\dint_0^{2\pi} t^2\sin 2t\;dt$
  \item $\dint \left(\tan^2 x + \tan^4 x\right)\;dx$
  \item $\dint \cot x \cos^2 x\;dx$
  \item $\dint \sec^4\theta \tan\theta\;d\theta$
\end{enumerate}
\newpage
\subsection*{Answers}
Below are answers to the problems on the reverse side of this page:
\begin{enumerate}
  \item $\dfrac{3}{2} e^{2x}\left(x - \dfrac{1}{2}\right) + C$
  \item $\dfrac{1}{6}\sin^6 y + C$
  \item $\dfrac{1}{2}(1 + e^2)$
  \item $1 - \dfrac{\pi}{4}$
  \item $\dfrac{2}{3} x\sec 3x - \dfrac{2}{9}\ln\left|\sec 3x + \tan 3x\right| + C$
  \item $\dfrac{1}{2}x - \dfrac{1}{4}\sin 2x + C$
  \item $-2\pi^2$
  \item $\dfrac{1}{3}\tan^3 x + C$
  \item $\ln\left|\sin x\right| - \dfrac{1}{2}\sin^2 x + C$
  \item $\dfrac{1}{4}\sec^4\theta + C$
\end{enumerate}

\end{document}
