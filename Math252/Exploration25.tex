\documentclass[12pt,letterpaper,fleqn]{article}

%       amslatex provides nice math extensions for typesetting mathematics
\usepackage{amsmath}
\usepackage{amsfonts}
\usepackage{tmmaths}
\usepackage{sympytex}

%       pstricks provides powerful environments for incorporating postscript into a
%       TeX/LaTeX document. You must have a postscript printer and a package like
%       dvips to convert the DVI file to a PS file.
%\usepackage{pst-all}
%\usepackage{pstricks,pst-plot}
%\usepackage{pst-coil,pst-node}

%  This package provides native tex support for numbered grids. The syntax is:
%  \graphpaper[spc](x_lowleft,y_lowleft)(x_upperright,y_upperright)

%\usepackage{graphpap}
%\usepackage{float}

%  The package below must be initialized with "\initfloatingfigs" immediately after the
%  "\begin{document} command.
%\usepackage{floatfig}

\usepackage{graphicx}
\graphicspath{{i:/mytex/graphics}}
\DeclareGraphicsExtensions{.ps,.eps}

%       tst is a package for the creation of exams, quizzes and tests. the include
%       file mathstuf (see below) provides many abbreviations for these environments.
%\usepackage{tst}

%       epsfig is a package which provides for the inclusion of Encapsulated PostScript
%       files in a document.
%\usepackage{epsfig}
%\usepackage{epic,eepic}
\include{mathstuf}
\usepackage[total={7.25in,10in},top=0.25in,left=0.75in,includehead]{geometry}
\usepackage{fancyhdr}
\pagestyle{fancy}
\lhead{Math 252}
\rhead{\large Name\makebox[2in]{\hrulefill}}
\chead{\LARGE Exploration 25}
%\lfoot{\today}
\cfoot{}
%\rfoot{\thepage}
\renewcommand{\headrulewidth}{0.4pt}
\renewcommand{\footrulewidth}{0.4pt}
\setlength{\parindent}{0pt}
\setlength{\parskip}{2ex}

\newcounter{tf}[enumi]
\newenvironment{tf}[0]{\begin{list}%
{\alph{tf}. \makebox[5em]{True\hfill False}}%
{\usecounter{tf}\setlength{\labelwidth}{7em}%
\setlength{\leftmargin}{3.5cm}%
\setlength{\labelsep}{1cm}}}%
{\end{list}}

%\usepackage{epic,eepic}
\newcommand{\numline}{%
%\newcounter{mark}%
%\setcounter{mark}{-1}%
\setlength{\unitlength}{0.1in}%
\begin{picture}(0,0)%
\thicklines%
\put(0,0){\line(1,0){60}}%
\multiput(0,0)(10,0){7}{\line(0,-1){1}%
\makebox(0,-1.5)[t]{\arabic{mark}}\stepcounter{mark}}%
%
\thinlines%
\multiput(0,0)(5,0){12}{\line(0,-1){0.5}}%
\multiput(0,0)(1,0){60}{\line(0,-1){0.3}}%
%\put(-5,265){\makebox(0,0)[l]{{\bf cm}}}%
\end{picture}}%

\newcommand{\ds}{\displaystyle}
\usepackage{amsfonts}


\let\oldhat\hat
\renewcommand{\hat}[1]{\oldhat{\boldsymbol{\mathbf{#1}}}}
\newcommand{\lv}[1]{\ensuremath{\langle #1 \rangle}}
\renewcommand{\i}{\ensuremath{\hat{\imath}}}
\renewcommand{\j}{\ensuremath{\hat{\jmath}}}
\renewcommand{\k}{\ensuremath{\mathbf{\oldhat{k}}}}
\newcommand{\ora}[1]{\ensuremath{\overrightarrow{#1}}}
\renewcommand{\vec}[1]{\ensuremath{\mathbf{#1}}}
\renewcommand{\v}[1]{\ensuremath{\vec{#1}}}
\newcommand{\abs}[1]{\ensuremath{\lvert #1 \rvert}}

\usepackage{tabularx}
\usepackage{paralist}
\newcommand{\red}[1]{\textcolor{red}{#1}}
\newcommand{\blue}[1]{\textcolor{blue}{#1}}
% \newcommand{\ans}[1]{\quad\fbox{answer: \red{#1}}}
\newcommand{\ans}[1]{\mbox{{\bf Ans:} \blue{#1}}}
\newcommand{\dd}[2][]{\ensuremath{\frac{\text{d}#1}{\text{d}#2}}}
\newcommand{\eval}[2]{\ensuremath{\left.#1\right|_{#2}}}

\usepackage{amsthm}

\theoremstyle{definition}
\newtheorem*{definition}{Definition}

\usepackage{enumitem}
\usepackage{subfig}

\newcommand{\dint}{\ensuremath{\displaystyle\int}}

\usepackage{amsthm}
\newtheorem*{theorem}{Theorem}

\theoremstyle{defintion}
\newtheorem*{exmp}{Example}

\begin{document}
\section*{Series}
  A series $\sum_{i=1}^\infty a_i$ converges to a limit $L$ if and only if the sequence $s_n = \sum_{i=1}^n$ of partial sums converge to $L$. If the sequence of partial sums does not converge to a limiting value, the series diverges (does not have a sum).

  A necessary (but \emph{not} sufficient) condition for convergence is that the terms $a_i$ of the series converge to zero.

  \theorem[Divergence Theorem] If $a_i \not\to 0$, then $\sum a_i$ diverges.

  \emph{Be warned} that $a_i \to 0$ {\bf does not} imply that $\sum a_i$ converges! The terms of the harmonic series $\sum \frac{1}{i}$ converge to zero \emph{but} the series \emph{diverges to infinity}.

  \theorem[Geometric Series] A geometric series $\sum_{i=1}^\infty ar^{i-1}$ converges to a limit $\dfrac{a}{1-r}$ if $|r| < 1$ and diverges otherwise.

  For those values of $r$ (the ``common ratio'') for which the series converges, the limiting value is $\frac{\text{the first term of the series}}{1 - r}$

  \theoremstyle{definition}
  \begin{definition}[Telescoping Series]
    A ``telescoping series'' is a series for which most of the interior terms of its partial sums subtract out (cancel), leaving only a few terms from the beginning and end of the sum.
  \end{definition}
  \begin{exmp}
    Consider $\ds\sum_{i=1}^\infty \left(\frac{1}{i+1} - \frac{1}{i}\right)$. The $n$th partial sum $s_n$ is
    \begin{align*}
      s_n = \sum_{i=1}^n \left(\frac{1}{i+1} - \frac{1}{i}\right) &= \left(\frac{1}{2} - 1\right) + \left(\frac{1}{3} - \frac{1}{2}\right) + \left(\frac{1}{4} - \frac{1}{3}\right) +\cdots + \left(\frac{1}{n} - \frac{1}{n-1}\right) + \left(\frac{1}{n+1} - \frac{1}{n}\right)\\
      &= -1 + \frac{1}{n+1}
    \end{align*} Hence, $\lim_{n\to\infty} s_n = -1$ and so the series converges to $-1$.
  \end{exmp}
  \subsection*{Problems}
  \begin{enumerate}
    \item Determine if the series diverges or \emph{might possibly} converge.
    \begin{enumerate}
      \item $\ds\sum_{i=1}^\infty\sin(i)$
      \item $\ds\sum_{i=1}^\infty\sin(1/i)$
      \item $\ds\sum_{i=1}^\infty \frac{1}{1 + (2/3)^i}$
      \item $\ds\sum_{i=1}^\infty \frac{i^2}{i^3+1}$
    \end{enumerate}
    \newpage
    \item Determine if the \emph{geometric series} converges or diverges. If it converges, find its limiting value.
    \begin{enumerate}
      \item $\ds\sum_{i=1}^\infty 5\left(\frac{-3}{5}\right)^{i-1}$
      \item $\ds\sum_{i=1}^\infty 2\left(\frac{5}{3}\right)^{i-1}$
      \item $\ds\sum_{i=2}^\infty\frac{2\cdot 8^i}{3^{2i-1}}$
      \item $\ds\sum_{i=2}^\infty\frac{e^{2i}}{6^{i-1}}$
    \end{enumerate}
    \item For each \emph{telescoping series} write an expression for the $n$th partial sum $s_n$, cancelling out interior terms, then evaluate $\lim_{n\to\infty} s_n$ to determine if the series converges or diverges. If it converges, give its limiting value
    \begin{enumerate}
      \item $\ds\sum_{i=1}^\infty\left(\cos\frac{1}{i} - \cos\frac{1}{i+1}\right)$
      \item $\ds\sum_{i=1}^\infty\frac{2}{i^2+2i}$ Hint: Express the fraction as a sum of partial fractions.
    \end{enumerate}
  \end{enumerate}
\end{document}
