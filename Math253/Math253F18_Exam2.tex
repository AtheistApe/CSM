\documentclass[13pt,letterpaper,fleqn]{article}

%       amslatex provides nice math extensions for typesetting mathematics
\usepackage{amsmath}
\usepackage{amsfonts}
\usepackage{tmmaths}
\usepackage{sympytex}

%       pstricks provides powerful environments for incorporating postscript into a
%       TeX/LaTeX document. You must have a postscript printer and a package like
%       dvips to convert the DVI file to a PS file.
%\usepackage{pst-all}
%\usepackage{pstricks,pst-plot}
%\usepackage{pst-coil,pst-node}

%  This package provides native tex support for numbered grids. The syntax is:
%  \graphpaper[spc](x_lowleft,y_lowleft)(x_upperright,y_upperright)

%\usepackage{graphpap}
%\usepackage{float}

%  The package below must be initialized with "\initfloatingfigs" immediately after the
%  "\begin{document} command.
%\usepackage{floatfig}

\usepackage{graphicx}
\graphicspath{{i:/mytex/graphics}}
\DeclareGraphicsExtensions{.ps,.eps}

%       tst is a package for the creation of exams, quizzes and tests. the include
%       file mathstuf (see below) provides many abbreviations for these environments.
%\usepackage{tst}

%       epsfig is a package which provides for the inclusion of Encapsulated PostScript
%       files in a document.
%\usepackage{epsfig}
%\usepackage{epic,eepic}
\include{mathstuf}
\usepackage[total={7.25in,10in},top=0.25in,left=0.75in,includehead]{geometry}
\usepackage{fancyhdr}
\pagestyle{fancy}
\lhead{Math 253}
\rhead{\large Name\makebox[2in]{\hrulefill}}
\chead{\LARGE Exam 2}
%\lfoot{\today}
\cfoot{}
%\rfoot{\thepage}
\renewcommand{\headrulewidth}{0.4pt}
\renewcommand{\footrulewidth}{0.4pt}
\setlength{\parindent}{0pt}
\setlength{\parskip}{2ex}

\newcounter{tf}[enumi]
\newenvironment{tf}[0]{\begin{list}%
{\alph{tf}. \makebox[5em]{True\hfill False}}%
{\usecounter{tf}\setlength{\labelwidth}{7em}%
\setlength{\leftmargin}{3.5cm}%
\setlength{\labelsep}{1cm}}}%
{\end{list}}

%\usepackage{epic,eepic}
\newcommand{\numline}{%
%\newcounter{mark}%
%\setcounter{mark}{-1}%
\setlength{\unitlength}{0.1in}%
\begin{picture}(0,0)%
\thicklines%
\put(0,0){\line(1,0){60}}%
\multiput(0,0)(10,0){7}{\line(0,-1){1}%
\makebox(0,-1.5)[t]{\arabic{mark}}\stepcounter{mark}}%
%
\thinlines%
\multiput(0,0)(5,0){12}{\line(0,-1){0.5}}%
\multiput(0,0)(1,0){60}{\line(0,-1){0.3}}%
%\put(-5,265){\makebox(0,0)[l]{{\bf cm}}}%
\end{picture}}%

\newcommand{\ds}{\displaystyle}
\usepackage{amsfonts}


\let\oldhat\hat
\renewcommand{\hat}[1]{\oldhat{\boldsymbol{\mathbf{#1}}}}
\newcommand{\lv}[1]{\ensuremath{\left\langle #1 \right\rangle}}
\renewcommand{\i}{\ensuremath{\hat{\imath}}}
\renewcommand{\j}{\ensuremath{\hat{\jmath}}}
\renewcommand{\k}{\ensuremath{\mathbf{\oldhat{k}}}}
\newcommand{\ora}[1]{\ensuremath{\overrightarrow{#1}}}
\renewcommand{\vec}[1]{\ensuremath{\pmb{#1}}}
\renewcommand{\v}[1]{\ensuremath{\vec{#1}}}
\newcommand{\abs}[1]{\ensuremath{\lvert #1 \rvert}}
\renewcommand{\deg}{\ensuremath{{}^\circ}}

\usepackage{tabularx}
\usepackage{paralist}
\newcommand{\red}[1]{\textcolor{red}{#1}}
\newcommand{\blue}[1]{\textcolor{blue}{#1}}
% \newcommand{\ans}[1]{\quad\fbox{answer: \red{#1}}}
\newcommand{\ans}[1]{\mbox{{\bf Ans:} \blue{#1}}}
\newcommand{\dd}[2][]{\ensuremath{\frac{\text{d}#1}{\text{d}#2}}}
\newcommand{\eval}[2]{\ensuremath{\left.#1\right|_{#2}}}
\newcommand{\proj}[2]{\ensuremath{\text{proj}_{\v{#2}}\v{#1}}}

\usepackage{enumitem}

\begin{document}
Please work the problems below on the paper provided to you in a neat,
clear and complete manner. Make it easy for me to grade.
\begin{enumerate}
 \item An objects velocity $\vec{v}$ at time $t$ seconds is given by $\v{v}(t) = \lv{1, 1-\cos t, -\sin t}$, for $t > 0$ seconds.
       \begin{enumerate}
        \item Find the position function $\v{r}(t)$, given that $\v{r}(0) = \lv{0,0,0}$.
        \item Find the unit tangent vector $\v{\hat{T}}$ to $\v{r}(t)$ at time $t = \pi/2$.
        \item Find the accelleration function $\v{a}(t)$.
       \end{enumerate}
 \item $\v{r}(t) = \lv{2t, \frac{4}{3}t^{3/2}, \frac{1}{2}t^2}$, for $t \geq 0$, is the vector position function of a curve.
       \begin{enumerate}
        \item Find the arc length function $s(t)$, for $t \geq 0$.
        \item Find the length of the curve over the interval $0 \leq t \leq 6$. Hint: The answer is between 25 and 32.
        \item What are the coordinates of the point, six units along the curve
              from $\v{r}(0)$? Hint: Find $t$, when $s = 6$.
        \item Find the curvature, $\kappa$, at $t = 1$
       \end{enumerate}
\end{enumerate}
% \rule{\linewidth}{1pt}
% %\newpage
% In the formulas below, $\v{v}=\v{r}'$, $\v{a}=\v{r}''$, where $\v{r}(t)$ is the
% vector position function. All vectors are functions of $t$, since $\v{r}(t)$ is a
% function of $t$.
% \begin{itemize}
%  \item $\vec{T} = \dfrac{\vec{v}}{|\vec{v}|} \qquad \vec{N} = \dfrac{\vec{T}'}{|\vec{T}'|}
%         \qquad \vec{B} = \vec{T} \times \vec{N}$.
%  \item $\kappa =  \dfrac{|\v{T}'|}{|\v{v}|} = \dfrac{|\vec{v} \times
%          \vec{a}|}{|\vec{v}|^3}$
%  \item $\v{a} = a_{T}\v{T} + a_{N}\v{N}$, where:
%        $\ds a_{T} = \frac{d}{dt} |\v{v}| = \frac{\v{v} \cdot \v{a}}{|\v{v}|}$
%        \;\; and \;\;
%        $\ds a_{N} = \kappa |\v{v}|^2 = \frac{|\v{v} \times \v{a}|}{|\v{v}|}$.
%  \item $\ds L = \int_a^b |\vec{v}(t)|\;dt$.
%  \item $\ds s(t) = \int_a^t |\vec{v}(u)|\;du$.
%  \item $\text{grad}(f) = \nabla f$
%  \item $\text{div}(\vec{F}) = \nabla\cdot\vec{F}$
%  \item $\text{curl}(\vec{F}) = \nabla\times\vec{F}$
% \end{itemize}
\end{document}
