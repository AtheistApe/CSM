\documentclass[12pt,letterpaper,fleqn]{article}

%       amslatex provides nice math extensions for typesetting mathematics
\usepackage{amsmath}
\usepackage{amsfonts}
\usepackage{tmmaths}
\usepackage{sympytex}

%       pstricks provides powerful environments for incorporating postscript into a
%       TeX/LaTeX document. You must have a postscript printer and a package like
%       dvips to convert the DVI file to a PS file.
%\usepackage{pst-all}
%\usepackage{pstricks,pst-plot}
%\usepackage{pst-coil,pst-node}

%  This package provides native tex support for numbered grids. The syntax is:
%  \graphpaper[spc](x_lowleft,y_lowleft)(x_upperright,y_upperright)

%\usepackage{graphpap}
%\usepackage{float}

%  The package below must be initialized with "\initfloatingfigs" immediately after the
%  "\begin{document} command.
%\usepackage{floatfig}

\usepackage{graphicx}
\graphicspath{{i:/mytex/graphics}}
\DeclareGraphicsExtensions{.ps,.eps}

%       tst is a package for the creation of exams, quizzes and tests. the include
%       file mathstuf (see below) provides many abbreviations for these environments.
%\usepackage{tst}

%       epsfig is a package which provides for the inclusion of Encapsulated PostScript
%       files in a document.
%\usepackage{epsfig}
%\usepackage{epic,eepic}
\include{mathstuf}
\usepackage[total={7.25in,10in},top=0.25in,left=0.75in,includehead]{geometry}
\usepackage{fancyhdr}
\pagestyle{fancy}
\lhead{Math 253}
\rhead{\large Name\makebox[2in]{\hrulefill}}
\chead{\LARGE Exploration 14}
%\lfoot{\today}
\cfoot{}
%\rfoot{\thepage}
\renewcommand{\headrulewidth}{0.4pt}
\renewcommand{\footrulewidth}{0.4pt}
\setlength{\parindent}{0pt}
\setlength{\parskip}{2ex}

\newcounter{tf}[enumi]
\newenvironment{tf}[0]{\begin{list}%
{\alph{tf}. \makebox[5em]{True\hfill False}}%
{\usecounter{tf}\setlength{\labelwidth}{7em}%
\setlength{\leftmargin}{3.5cm}%
\setlength{\labelsep}{1cm}}}%
{\end{list}}

%\usepackage{epic,eepic}
\newcommand{\numline}{%
%\newcounter{mark}%
%\setcounter{mark}{-1}%
\setlength{\unitlength}{0.1in}%
\begin{picture}(0,0)%
\thicklines%
\put(0,0){\line(1,0){60}}%
\multiput(0,0)(10,0){7}{\line(0,-1){1}%
\makebox(0,-1.5)[t]{\arabic{mark}}\stepcounter{mark}}%
%
\thinlines%
\multiput(0,0)(5,0){12}{\line(0,-1){0.5}}%
\multiput(0,0)(1,0){60}{\line(0,-1){0.3}}%
%\put(-5,265){\makebox(0,0)[l]{{\bf cm}}}%
\end{picture}}%

\newcommand{\ds}{\displaystyle}
\usepackage{amsfonts}


\let\oldhat\hat
\renewcommand{\hat}[1]{\oldhat{\boldsymbol{\mathbf{#1}}}}
\newcommand{\lv}[1]{\ensuremath{\left\langle #1 \right\rangle}}
\renewcommand{\i}{\ensuremath{\hat{\imath}}}
\renewcommand{\j}{\ensuremath{\hat{\jmath}}}
\renewcommand{\k}{\ensuremath{\mathbf{\oldhat{k}}}}
\newcommand{\ora}[1]{\ensuremath{\overrightarrow{#1}}}
\renewcommand{\vec}[1]{\ensuremath{\pmb{#1}}}
\renewcommand{\v}[1]{\ensuremath{\vec{#1}}}
\newcommand{\abs}[1]{\ensuremath{\lvert #1 \rvert}}
\renewcommand{\deg}{\ensuremath{{}^\circ}}

\usepackage{tabularx}
\usepackage{paralist}
\newcommand{\red}[1]{\textcolor{red}{#1}}
\newcommand{\blue}[1]{\textcolor{blue}{#1}}
% \newcommand{\ans}[1]{\quad\fbox{answer: \red{#1}}}
\newcommand{\ans}[1]{\mbox{{\bf Ans:} \blue{#1}}}
\newcommand{\dd}[2][]{\ensuremath{\frac{\text{d}#1}{\text{d}#2}}}
\newcommand{\eval}[2]{\ensuremath{\left.#1\right|_{#2}}}
\newcommand{\proj}[2]{\ensuremath{\text{proj}_{\v{#2}}\v{#1}}}

\usepackage{enumitem}

\begin{document}
\section*{Arclength and the Arclength function}
\subsection*{Arclength}
Imagine a car moving at a \emph{constant} speed $\|\vec{v}(t)\|$ along a curvy road. The distance it travels during a time interval $\Delta t$ is $\|\vec{v}(t)\|\Delta t$. This distance measures the length of road $\Delta s$ travelled during the time interval $\Delta t$.\\[1.5ex] Now imagine the car moving at a \emph{continuously varying} speed $\|\vec{v}(t)\|$ over a time interval $a \leq t \leq b$. Over a very small time interval $dt$ at some particular time $t$ within this interval, the car's speed is the (approximately) constant speed $\|\vec{v}(t)\|$. Therefore, the distance $ds$ it travels over this time interval is $\|\vec{v}(t)\| dt$. Summing up (integrating) all these distances over the whole time interval $[a, b]$, we get the length of road travelled over this time interval.
  \begin{equation*}
    s = \int_a^b \|\vec{v}(t)\|\;dt
  \end{equation*}
  where $s$ is the length of road (arc-length of the curve) travelled over the time interval $[a, b]$.
\subsubsection*{Problems}
  \begin{enumerate}
    \item Let $\vec{r}(t) = \lv{t, -\frac{3}{4} t + 3}$, for $0\leq t\leq 4$ be the straight line segment $L$ from $(0,3)$ to $(4,0)$.
    \begin{enumerate}
      \item Using geometry, find the length of $L$.
      \item Recalling that $\vec{v}(t) = \vec{r}'(t)$, find the length of $L$ using the arclength integral above. Do the two answers agree?
    \end{enumerate}
    \item The same straight line segment $L$ from $(0,3)$ to $(4,0)$ can also be described by the vector-valued function $\vec{r}(t) = \lv{t^2/4, -3t^2/16 + 3}$, for $0\leq t\leq 4$ (how can you verify this?) With this parameterization for $L$, what is the value of the arclength integral? Does this answer agree with previous answers for the length of $L$? If not, why not?
    \item The same straight line segment $L$ from $(0,3)$ to $(4,0)$ can also be described by the vector-valued function $\vec{r}(t) = \lv{4\sin t, -3\sin t + 3}$, for $0\leq t\leq \pi/2$ (how can you verify this?) With this parameterization for $L$, what is the value of the arclength integral? Does this answer agree with previous answers for the length of $L$? If not, why not?
    \item The same straight line segment $L$ from $(0,3)$ to $(4,0)$ can also be described by the vector-valued function $\vec{r}(t) = \lv{4\sin t, -3\sin t + 3}$, for $0\leq t\leq \pi$ (how can you verify this?) With this parameterization for $L$, what is the value of the arclength integral? Does this answer agree with previous answers for the length of $L$? If not, why not?
  \end{enumerate}
  \newpage
  \subsection*{Arclength function}
  Let $\vec{r}(t)$ for $a\leq t\leq b$ define a curve $C$. We define the \emph{arclength function} $s(t)$, $a\leq t\leq b$ by the integral
  \begin{equation*}
    s(t) = \int_a^t \|\vec{r}'(\tau)\|\;d\tau
  \end{equation*}
  This function gives us the distance travelled along the curve $C$ (the length of the curve) over the time interval $[a, t]$.
  \subsubsection*{Problems}
  \begin{enumerate}
    \item Let $\vec{r}(t) = \lv{t, -3t/4 + 3}$, for $0\leq t\leq 4$ be the straight line segment $L$ from $(0,3)$ to $(4,0)$. Find $s(t)$, then evaluate $s(1)$, $s(2)$, $s(3)$, and $s(4)$.
    \item The same straight line segment $L$ from $(0,3)$ to $(4,0)$ can also be described by the vector-valued function $\vec{r}(t) = \lv{t^2/4, -3t^2/16 + 3}$, for $0\leq t\leq 4$. Find $s(t)$, then evaluate $s(1)$, $s(2)$, $s(3)$, and $s(4)$. How do these distances compare with the distances in the previous problem at the same times?
  \end{enumerate}
\end{document}
Curve(m*abs(v) cos(θ)/k (1 - ℯ^(-k t / m)), m / k ((abs(v) sin(θ) + m g / k) (1 - ℯ^(-k t / m)) - g t), t, 0, 2π)
