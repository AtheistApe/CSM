\documentclass[14pt,letterpaper,fleqn]{article}

%       amslatex provides nice math extensions for typesetting mathematics
\usepackage{amsmath}
\usepackage{amsfonts}
\usepackage{tmmaths}
\usepackage{sympytex}

%       pstricks provides powerful environments for incorporating postscript into a
%       TeX/LaTeX document. You must have a postscript printer and a package like
%       dvips to convert the DVI file to a PS file.
%\usepackage{pst-all}
%\usepackage{pstricks,pst-plot}
%\usepackage{pst-coil,pst-node}

%  This package provides native tex support for numbered grids. The syntax is:
%  \graphpaper[spc](x_lowleft,y_lowleft)(x_upperright,y_upperright)

%\usepackage{graphpap}
%\usepackage{float}

%  The package below must be initialized with "\initfloatingfigs" immediately after the
%  "\begin{document} command.
%\usepackage{floatfig}

\usepackage{graphicx}
\graphicspath{{i:/mytex/graphics}}
\DeclareGraphicsExtensions{.ps,.eps}

%       tst is a package for the creation of exams, quizzes and tests. the include
%       file mathstuf (see below) provides many abbreviations for these environments.
%\usepackage{tst}

%       epsfig is a package which provides for the inclusion of Encapsulated PostScript
%       files in a document.
%\usepackage{epsfig}
%\usepackage{epic,eepic}
\include{mathstuf}
\usepackage[total={7.25in,10in},top=0.25in,left=0.75in,includehead]{geometry}
\usepackage{fancyhdr}
\pagestyle{fancy}
\lhead{Math 253}
\rhead{\large Name\makebox[2in]{\hrulefill}}
\chead{\LARGE Exam Review}
%\lfoot{\today}
\cfoot{}
%\rfoot{\thepage}
\renewcommand{\headrulewidth}{0.4pt}
\renewcommand{\footrulewidth}{0.4pt}
\setlength{\parindent}{0pt}
\setlength{\parskip}{2ex}

\newcounter{tf}[enumi]
\newenvironment{tf}[0]{\begin{list}%
{\alph{tf}. \makebox[5em]{True\hfill False}}%
{\usecounter{tf}\setlength{\labelwidth}{7em}%
\setlength{\leftmargin}{3.5cm}%
\setlength{\labelsep}{1cm}}}%
{\end{list}}

%\usepackage{epic,eepic}
\newcommand{\numline}{%
%\newcounter{mark}%
%\setcounter{mark}{-1}%
\setlength{\unitlength}{0.1in}%
\begin{picture}(0,0)%
\thicklines%
\put(0,0){\line(1,0){60}}%
\multiput(0,0)(10,0){7}{\line(0,-1){1}%
\makebox(0,-1.5)[t]{\arabic{mark}}\stepcounter{mark}}%
%
\thinlines%
\multiput(0,0)(5,0){12}{\line(0,-1){0.5}}%
\multiput(0,0)(1,0){60}{\line(0,-1){0.3}}%
%\put(-5,265){\makebox(0,0)[l]{{\bf cm}}}%
\end{picture}}%

\newcommand{\ds}{\displaystyle}
\usepackage{amsfonts}


\let\oldhat\hat
\renewcommand{\hat}[1]{\oldhat{\boldsymbol{\mathbf{#1}}}}
\newcommand{\lv}[1]{\ensuremath{\left\langle #1 \right\rangle}}
\renewcommand{\i}{\ensuremath{\hat{\imath}}}
\renewcommand{\j}{\ensuremath{\hat{\jmath}}}
\renewcommand{\k}{\ensuremath{\mathbf{\oldhat{k}}}}
\newcommand{\ora}[1]{\ensuremath{\overrightarrow{#1}}}
\renewcommand{\vec}[1]{\ensuremath{\pmb{#1}}}
\renewcommand{\v}[1]{\ensuremath{\vec{#1}}}
\newcommand{\abs}[1]{\ensuremath{\lvert #1 \rvert}}
\renewcommand{\deg}{\ensuremath{{}^\circ}}

\usepackage{tabularx}
\usepackage{paralist}
\newcommand{\red}[1]{\textcolor{red}{#1}}
\newcommand{\blue}[1]{\textcolor{blue}{#1}}
% \newcommand{\ans}[1]{\quad\fbox{answer: \red{#1}}}
\newcommand{\ans}[1]{\mbox{{\bf Ans:} \blue{#1}}}
\newcommand{\dd}[2][]{\ensuremath{\frac{\text{d}#1}{\text{d}#2}}}
\newcommand{\eval}[2]{\ensuremath{\left.#1\right|_{#2}}}
\newcommand{\proj}[2]{\ensuremath{\text{proj}_{\v{#2}}\v{#1}}}

\usepackage{enumitem}

\begin{document}
\section*{Extra Review Problems for Exam 2}
\subsection*{Problems}
\begin{enumerate}
 \item Find a vector function for the line tangent to the curve $\vec{r}(t) = \lv{\cos t, \sin t, \cos 4t}$ at $t = \pi/3$.
 \item An object moves with velocity vector $\vec{v} = \lv{t, t^2, \cos t}$, starting at $\lv{0,0,0}$ at $t = 0$. Find the position function $\vec{r}(t)$.
 \item The position function of a particle is given by $\vec{r}(t) = \lv{t^2, 5t, t^2-16}$, for $t\geq 0$. When (at what value of $t$) is the speed of the particle a minimum?
 \item A curve is given by the position function $\vec{r}(t) = \lv{t^2, 2, t^3}$. Find the length of the curve over the interval $0\leq t\leq 1$.
 \item Find the curvature of the curve $\vec{r}(t)=\lv{t, t^2, t}$ at time $t = 1$.
\end{enumerate}
\subsection*{Answers}
\begin{enumerate}
 \item The tangent line to the curve $\vec{r}(t)$ at $t=\pi/3$ is given by $\v{L}(t) = \lv{\dfrac{1}{2} - \dfrac{t\sqrt{3}}{2}, \dfrac{\sqrt{3}}{2} + \dfrac{t}{2}, -\dfrac{1}{2} + 2t\sqrt{2}}$
 \item $\v{r}(t) = \lv{\dfrac{1}{2}t^2, \dfrac{1}{3}t^3, \sin t}$
 \item The speed function is $\sqrt{8t^2 + 25}$. Clearly, by inspection, its minimum value occurs at $t = 0$. The minumum speed is $5$.
 \item The length of the curve $\v{r}(t)$ over the interval $[0,1]$ is $s = \ds\int_0^1 t\sqrt{4 + 9t^2}\;dt = \dfrac{13\sqrt{13} - 8}{27}\approx 1.44$
 \item The curvature of $\v{r}(t)$ at time $t = 1$ is $\kappa(1) = \dfrac{\sqrt{3}}{9}\approx 0.192$
\end{enumerate}
\end{document}
