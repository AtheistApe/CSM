%        File: exam1.tex
%     Created: Thu Feb 07 11:00 PM 2013 P
% Last Change: Thu Feb 07 11:00 PM 2013 P
%
\documentclass[12pt,letterpaper,fleqn]{article}

%       amslatex provides nice math extensions for typesetting mathematics
\usepackage{amsmath}
\usepackage{tmmaths}

%       pstricks provides powerful environments for incorporating postscript into a
%       TeX/LaTeX document. You must have a postscript printer and a package like
%       dvips to convert the DVI file to a PS file.
%\usepackage{pst-all}
%\usepackage{pstricks,pst-plot}
%\usepackage{pst-coil,pst-node}

%   tikz provides high-level language constructs for creating graphics using pgf.
\usepackage{tikz}
%\usetikzlibrary{arrows,snakes,backgrounds}

%  This package provides native tex support for numbered grids. The syntax is:
%  \graphpaper[spc](x_lowleft,y_lowleft)(x_upperright,y_upperright)

%\usepackage{graphpap}
%\usepackage{float}

%  The package below must be initialized with "\initfloatingfigs" immediately after the
%  "\begin{document} command.
%\usepackage{floatfig}

%\usepackage{graphicx}
%\graphicspath{{i:/mytex/graphics}}
%\DeclareGraphicsExtensions{.ps,.eps}

%       tst is a package for the creation of exams, quizzes and tests. the include
%       file mathstuf (see below) provides many abbreviations for these environments.
%\usepackage{tst}

%       epsfig is a package which provides for the inclusion of Encapsulated PostScript
%       files in a document.
%\usepackage{epsfig}
%\usepackage{epic,eepic}
\usepackage[total={7.25in,10in},top=0.25in,left=0.75in,includehead]{geometry}
\usepackage{fancyhdr}
\pagestyle{fancy}
\lhead{Math 253}
\rhead{\large Name\makebox[2in]{\hrulefill}}
\chead{\LARGE Exam 1}
\lfoot{\today}
\cfoot{}
%\rfoot{\thepage}
\renewcommand{\headrulewidth}{0.4pt}
\renewcommand{\footrulewidth}{0.4pt}
\setlength{\parindent}{0pt}
\setlength{\parskip}{2ex}

%\newcounter{tf}[enumi]
%\newenvironment{tf}[0]{\begin{list}%
%{\alph{tf}. \makebox[5em]{True\hfill False}}%
%{\usecounter{tf}\setlength{\labelwidth}{7em}%
%\setlength{\leftmargin}{3.5cm}%
%\setlength{\labelsep}{1cm}}}%
%{\end{list}}

%\usepackage{epic,eepic}
%\newcommand{\numline}{%
%\newcounter{mark}%
%\setcounter{mark}{-1}%
%\setlength{\unitlength}{0.1in}%
%\begin{picture}(0,0)%
%\thicklines%
%\put(0,0){\line(1,0){60}}%
%\multiput(0,0)(10,0){7}{\line(0,-1){1}%
%\makebox(0,-1.5)[t]{\arabic{mark}}\stepcounter{mark}}%
%
%\thinlines%
%\multiput(0,0)(5,0){12}{\line(0,-1){0.5}}%
%\multiput(0,0)(1,0){60}{\line(0,-1){0.3}}%
%\put(-5,265){\makebox(0,0)[l]{{\bf cm}}}%
%\end{picture}}%

\newcommand{\ds}{\displaystyle}
\renewcommand{\deg}{\ensuremath{{}^\circ}}
%\renewcommand{\vec}{\ensuremath{\mathbf}}
\renewcommand{\deg}{\ensuremath{{}^\circ}}
\let\oldhat\hat
\renewcommand{\hat}[1]{\oldhat{\boldsymbol{\mathbf{#1}}}}
\newcommand{\lv}[1]{\ensuremath{\langle #1 \rangle}}
\newcommand{\ih}{\ensuremath{\hat{\imath}}}
\newcommand{\jh}{\ensuremath{\hat{\jmath}}}
\newcommand{\kh}{\ensuremath{\mathbf{\oldhat{k}}}}
\newcommand{\blnk}{\rule[-0.5ex]{2ex}{1pt}}
\usepackage{tabularx}
\begin{document}
Please work the problems in a neat, clear manner. Show enough detail to
allow me to follow your solutions; Make it easy for me to grade.
\begin{enumerate}
 \item Let $\vec{a} = \lv{1, 1, -2} \text{ and } \vec{b} = \lv{9, 2 , -6}$.
       Find:
       \begin{enumerate}
        \item $\vec{a}\cdot\vec{b}$
        \item $\vec{a}\times\vec{b}$
        \item $\|\vec{b}\|$
        \item $\hat{b}$ (a unit vector in the direction of $\vec{b}$.
        \item The angle (to the nearest degree) between $\vec{a}$ and $\vec{b}$.
        \item The area of the parallelogram that can be formed by vectors $\vec{a}$ and $\vec{b}$.
              % \item $\text{comp}_{\vec{b}} \vec{a}$
        \item $\text{proj}_{\vec{a}} \vec{b}$
       \end{enumerate}
 \item Given vectors $\vec{a}$ and $\vec{b}$ such that $\|\vec{a}\| = 3$, $\|\vec{b}\| = 4$ and $\vec{a}\cdot\vec{b} = 6\sqrt{3}$, find $\|(\vec{a} + 2\vec{b})\times (3\vec{a} - \vec{b})\|$.
 \item Find all values of $b$ such that the vectors $\lv{-6,b,2}$ and
       $\lv{b,b^2,b}$ are orthogonal.
 \item Find the parametric and symmetric equations of the line that goes through the point $P(-2,3,5)$ and is parallel to the line with vector equation $\vec{r}(t)=\lv{1-t, 2+t, -4-4t}$.
       % \item Find parametric equations of the line through the point $P(4,-1,2)$
       %   and $(1,1,5)$.
       % \item Find an equation of the plane which contains the point $(4,4,1)$ and
       %   is perpendicular to the line $\mathbf{L}(t) = \lv{1 - t, -1 + 4t, 9 - 8t}$.
 \item Find an equation ($ax + by + cz = d$) of the plane that contains the point $(-2,1,3)$ and is perpendicular to the line through the points $(-2,1,3)$ and $(1,0,-1)$.
       % \item Given two intersecting planes with equations $x - 2y + z = 1$ and
       %       $2x + y - z = 5$,
       %       \begin{enumerate}
       %        \item Find the coordinates of a point on the line of intersection of the two planes. Hint: The coordinates of the point must satisfy both equations.
       %        \item The angle of intersection of the two planes is defined to be the angle formed by normal vectors of the planes. Find the angle of intersection of the two planes.
       %        \item Find the vector equation of the line of intersection of the two planes. Hints: Every line in a plane is orthogonal to a normal vector of the plane. The line of intersection is in both planes.
       %       \end{enumerate}
\end{enumerate}
\end{document}

%   \item
%   \begin{enumerate}
%     \begin{tabularx}{\linewidth}{p{3in} p{3in}}
%     \item  &
%     \item   \\[2in]
%     \end{tabularx}
%   \end{enumerate}


% \begin{minipage}[t]{0.5\textwidth}
%   \vspace{0pt}
% \item
%   \begin{enumerate}
%   \end{enumerate}
% \end{minipage}
% \hfill
% \begin{minipage}[t]{0.35\textwidth}
%   \vspace{0pt}
%   \includegraphics[width=\textwidth]{d:/}%
% \end{minipage}


% \begin{minipage}[t]{0.45\linewidth}
% \vspace{0pt}
% %\begin{center}
%   \psset{unit=0.3in}
%   \pspicture*[0](-5.5,-5.5)(5.5,5.5)
%   \psgrid[gridwidth=0.4pt,subgridcolor=black,subgriddots=5,subgriddiv=4,gridlabels=0pt]
% %  \psaxes*[Dx=1,Dy=1,dx=1,dy=1]{<->}(0,0)(-6,-6)(6,6)
%   \endpspicture
% %\end{center}
% \end{minipage}


% \begin{minipage}[t]{6in}
% \begin{picture}(0,0)(6,0.75)
% \setlength{\unitlength}{1in}
% \newcounter{mark}%
% \setcounter{mark}{-5}%
% \put(0,0.25){\numline}
% \put(4.7,0.25){\circle*{0.1}}
% \put(4.7,0.35){$A$}
% \end{picture}
% \end{minipage}
