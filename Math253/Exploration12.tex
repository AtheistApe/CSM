\documentclass[12pt,letterpaper,fleqn]{article}

%       amslatex provides nice math extensions for typesetting mathematics
\usepackage{amsmath}
\usepackage{amsfonts}
\usepackage{tmmaths}
\usepackage{sympytex}

%       pstricks provides powerful environments for incorporating postscript into a
%       TeX/LaTeX document. You must have a postscript printer and a package like
%       dvips to convert the DVI file to a PS file.
%\usepackage{pst-all}
%\usepackage{pstricks,pst-plot}
%\usepackage{pst-coil,pst-node}

%  This package provides native tex support for numbered grids. The syntax is:
%  \graphpaper[spc](x_lowleft,y_lowleft)(x_upperright,y_upperright)

%\usepackage{graphpap}
%\usepackage{float}

%  The package below must be initialized with "\initfloatingfigs" immediately after the
%  "\begin{document} command.
%\usepackage{floatfig}

\usepackage{graphicx}
\graphicspath{{i:/mytex/graphics}}
\DeclareGraphicsExtensions{.ps,.eps}

%       tst is a package for the creation of exams, quizzes and tests. the include
%       file mathstuf (see below) provides many abbreviations for these environments.
%\usepackage{tst}

%       epsfig is a package which provides for the inclusion of Encapsulated PostScript
%       files in a document.
%\usepackage{epsfig}
%\usepackage{epic,eepic}
\include{mathstuf}
\usepackage[total={7.25in,10in},top=0.25in,left=0.75in,includehead]{geometry}
\usepackage{fancyhdr}
\pagestyle{fancy}
\lhead{Math 253}
\rhead{\large Name\makebox[2in]{\hrulefill}}
\chead{\LARGE Exploration 12}
%\lfoot{\today}
\cfoot{}
%\rfoot{\thepage}
\renewcommand{\headrulewidth}{0.4pt}
\renewcommand{\footrulewidth}{0.4pt}
\setlength{\parindent}{0pt}
\setlength{\parskip}{2ex}

\newcounter{tf}[enumi]
\newenvironment{tf}[0]{\begin{list}%
{\alph{tf}. \makebox[5em]{True\hfill False}}%
{\usecounter{tf}\setlength{\labelwidth}{7em}%
\setlength{\leftmargin}{3.5cm}%
\setlength{\labelsep}{1cm}}}%
{\end{list}}

%\usepackage{epic,eepic}
\newcommand{\numline}{%
%\newcounter{mark}%
%\setcounter{mark}{-1}%
\setlength{\unitlength}{0.1in}%
\begin{picture}(0,0)%
\thicklines%
\put(0,0){\line(1,0){60}}%
\multiput(0,0)(10,0){7}{\line(0,-1){1}%
\makebox(0,-1.5)[t]{\arabic{mark}}\stepcounter{mark}}%
%
\thinlines%
\multiput(0,0)(5,0){12}{\line(0,-1){0.5}}%
\multiput(0,0)(1,0){60}{\line(0,-1){0.3}}%
%\put(-5,265){\makebox(0,0)[l]{{\bf cm}}}%
\end{picture}}%

\newcommand{\ds}{\displaystyle}
\usepackage{amsfonts}


\let\oldhat\hat
\renewcommand{\hat}[1]{\oldhat{\boldsymbol{\mathbf{#1}}}}
\newcommand{\lv}[1]{\ensuremath{\langle #1 \rangle}}
\renewcommand{\i}{\ensuremath{\hat{\imath}}}
\renewcommand{\j}{\ensuremath{\hat{\jmath}}}
\renewcommand{\k}{\ensuremath{\mathbf{\oldhat{k}}}}
\newcommand{\ora}[1]{\ensuremath{\overrightarrow{#1}}}
\renewcommand{\vec}[1]{\ensuremath{\pmb{#1}}}
\renewcommand{\v}[1]{\ensuremath{\vec{#1}}}
\newcommand{\abs}[1]{\ensuremath{\lvert #1 \rvert}}
\renewcommand{\deg}{\ensuremath{{}^\circ}}

\usepackage{tabularx}
\usepackage{paralist}
\newcommand{\red}[1]{\textcolor{red}{#1}}
\newcommand{\blue}[1]{\textcolor{blue}{#1}}
% \newcommand{\ans}[1]{\quad\fbox{answer: \red{#1}}}
\newcommand{\ans}[1]{\mbox{{\bf Ans:} \blue{#1}}}
\newcommand{\dd}[2][]{\ensuremath{\frac{\text{d}#1}{\text{d}#2}}}
\newcommand{\eval}[2]{\ensuremath{\left.#1\right|_{#2}}}
\newcommand{\proj}[2]{\ensuremath{\text{proj}_{\v{#2}}\v{#1}}}

\usepackage{enumitem}

\begin{document}
\section*{Lines and Planes in $\pmb{\mathbb{R}^3}$}
\subsection*{Lines}
Find parametric and symmetric equations of lines in $\mathbb{R}^3$ satisfying the following properties:
\begin{enumerate}
  \item The line passes through the point $P_0(-1,1,2)$ and is parallel to the vector $\v{v}=\lv{2,-1,3}$.
  \item The line contains the points $P(2,-3,3)$ and $Q(-1,0,5)$.
  \item The line passes through the point $P(1,2,3)$ and is perpendicular to the plane with equation $x + 2y - z = 1$
\end{enumerate}
\subsection*{Planes}
Find equations of planes ($ax + by + cz = d$) in $\mathbb{R}^3$ satisfying the following properties:
\begin{enumerate}[resume]
  \item The point $P_0(-1,3,-2)$ is on the plane and the plane is perpendicular to the vector $\v{n} = \lv{2,1,-3}$.
  \item The point $P_0(2,-5,1)$ is on the plane and the plane is perpendicular to the line that goes through the points $P(1,1,-1)$ and $Q(-3,2,3)$.
  \item The points $P(2,-1,4)$, $Q(1,1,1)$ and $R(-3,1,-2)$ are all on the plane. Hint: Find two vectors in the plane and use them to create a normal vector to the plane.
\end{enumerate}
\subsection*{Other Problems}
\begin{enumerate}[resume]
  \item Determine whether the lines $\lv{1,2,-1} + t\lv{1,2,3}$ and $\lv{1,0,1} + t\lv{2/3, 2, 4/3}$ are parallel, intersect or neither.
  \item Draw a sketch of a plane in $\mathbb{R}^3$ and a point $P$ not on the plane. Assuming you know the coordinates of some point $P_0$ on the plane and a normal (perpendicular) vector $\v{v}$ to the plane, describe how you can use vector projection to find the (perpendicular) distance from the point $P$ to the plane.
\end{enumerate}


\end{document}
